\subsubsection{Barycentric Coordinates of an Edge Projection}

In the case of two vertices of the  tetrahedron  being  located  infinitely  far
away, any 3D point located inside its volume will  be  projected  onto  the edge
formed by the two  finite  vertex  locations.  Thus, a projection matrix must be
constructed.

The edge of  a tetrahedron is defined by the two vertices $v_1$ and $v_2$, where
$v_n    \in    \mathbb{R}^3$.    The   3   dimensional   Cartesian    coordinate
$\vec{x}\in\mathbb{R}^3$  is projected onto one of the tetrahedron's edges using
equation  \ref{eq:projection}  where  the  matrix  $A$   is   defined  as  $A  =
\begin{bmatrix} \vec{v_2} - \vec{v_1} \end{bmatrix}$.

The matrices $B$ and $\text{proj}_V$ from equations \ref{eq:bary:cartesian}  and
\ref{eq:projection} may be combined to form the final matrix  for transforming a
3D coordinate $\vec{x}$ into projected
barycentric coordinates $\vec{\lambda}$:

\begin{align}
    \label{eq:bary:edge}
    \vec{\lambda} = TA\left(A\transpose A\right)\inverse A\transpose T\inverse B \vec{x}
\end{align}

In an effort to be more verbose, equation \ref{eq:bary:edge} is broken down  and
constructed  step-by-step  using  the  edge  vertices  $\vec{v_1}$, $\vec{v_2}$.

\begin{align*}
    \vec{a} &= \vec{v_2} - \vec{v_1} \\
    \text{proj}_V &= A\left(A\transpose A\right)\inverse A\transpose \\
    \begin{bmatrix}
        m_{00} & m_{01} & m_{02} \\
        m_{10} & m_{11} & m_{12} \\
        m_{20} & m_{21} & m_{22} \\
    \end{bmatrix}
    &=  \begin{bmatrix}
            a_x \\
            a_y \\
            a_z \\
        \end{bmatrix}
        \left(
            \begin{bmatrix}
                a_x & a_y & a_z \\
            \end{bmatrix}
            \begin{bmatrix}
                a_x \\
                a_y \\
                a_z \\
            \end{bmatrix}
        \right)\inverse
        \begin{bmatrix}
            a_x & a_y & a_z \\
        \end{bmatrix}
\end{align*}

The  calculated  projection matrix can be  sandwiched  between  the  translation
matrix  $T$  from equation \ref{eq:translate_offset} and then be multiplied with
the  barycentric transformation matrix $B$ from equation \ref{eq:bary:cartesian}
to yield the final transformation matrix.

\begin{align*}
    \vec{\lambda} = 
    \begin{bmatrix}
        1 & 0 & 0 & -v_{1_x} \\
        0 & 1 & 0 & -v_{1_y} \\
        0 & 0 & 1 & -v_{1_z} \\
        0 & 0 & 0 & 1 \\
    \end{bmatrix}
    \begin{bmatrix}
        m_{00} & m_{01} & m_{02} & 0 \\
        m_{10} & m_{11} & m_{12} & 0 \\
        m_{20} & m_{21} & m_{22} & 0 \\
        0      & 0      & 0      & 1 \\
    \end{bmatrix}
    \begin{bmatrix}
        1 & 0 & 0 & -v_{1_x} \\
        0 & 1 & 0 & -v_{1_y} \\
        0 & 0 & 1 & -v_{1_z} \\
        0 & 0 & 0 & 1 \\
    \end{bmatrix}\inverse
    \begin{bmatrix}
        v_{1_x} & v_{2_x} & v_{3_x} & v_{4_x} \\
        v_{1_y} & v_{2_y} & v_{3_y} & v_{4_y} \\
        v_{1_z} & v_{2_z} & v_{3_z} & v_{4_z} \\
        1 & 1 & 1 & 1 \\
    \end{bmatrix}
    \vec{x}
\end{align*}

